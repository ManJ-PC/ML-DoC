\chapter{Introduction} \label{chap:intro}

\section*{Context and Motivation} \label{sec:context} 
% with * not numerate the title

 \paragraph{}Consciousness is the individual's ability to be aware of the knowledge of self and the environment. Furthermore, it is the ability to respond to various voluntary internal and
external stimuli \citet{https://doi.org/10.1196/annals.1440.013}.
In basic neurological terms, it is composed of awareness and wakefulness \citet{zheng2017disentangling}.
The different states of consciousness are represented in the table \ref{tab: States} 
.

\begin{table}[!htb].

\begin{tabular}{|l|c|c|}
\hline
Condition                              & \multicolumn{1}{l|}{\textbf{Wakefulness}} & \multicolumn{1}{l|}{\textbf{Awareness}} \\ \hline
Coma                                   & -                                         & -                                       \\ \hline
\rowcolor[HTML]{FFFC9E} 
Vegetative State                       & + to ++                                   & -                                       \\ \hline
\rowcolor[HTML]{FFFC9E} 
Minimally Conscious State              & + to ++                                   & +                                       \\ \hline
Emerged from Minimally Conscious State & ++                                        & ++                                      \\ \hline
\end{tabular}%
\caption{Disorders of consciousness categorization}
\label{tab: States}
\end{table}
\vspace{0.2cm}
Brain Injury (BI) is a head injury that damages the brain and its complex connections \citep{HongSong2017}. 

This causes
problems with how a person can
think and interact with the world around
them. Following a BI, there are
specific cognitive skills that are no
longer functioning in the same capacity \cite{Bender2015}.
Perception, observation, and
recognition of information are deeply affected.

There are events that damage areas of the brain that control parts of the human body. And the patient's faculties are conditioned.
The origin of BI can be: 
\begin{itemize}
    \item acute: as in a virus or haemorrhage  \item traumatic: for example road traffic accidents, impacts where there are head injuries  \item non-traumatic: such as drowning, organ's infarction, drug overdose, etc.
\end{itemize} 
\paragraph{}All of the above cause brain damage that leads to consequences in terms of disturbing the person's consciousness \cite{teixeira2020disorders}.

After the coma, the rates of diagnostic errors, namely in the distinction between vegetative state (VS) and the minimally conscious state (MCS), are high  $\approx 40$\% \cite{andrews1996misdiagnosis}  \cite{ gill2004sensory} \cite{schnakers2009diagnostic}. 

The diagnosis of the patient who suffers brain damage is made using scales assessing the behavioural response to stimuli:
CRS-r, SMART, WHIM, WNSSP, Rancho levels physicians, etc. \citet{Gill-Thwaites2004}. \label{behscales}
But also with technological advancement such as neuronal imaging technologies such including:
   functional magnetic resonance imaging (fMRI),
   electroencephalogram (EEG)
   positron emission tomography (PET)   \cite{King2011} \citet{Da2019} \cite{L2018}.

  It has been possible to leverage the expertise in the area of neuroscience and the identification, consequent diagnosis, of the cerebral behaviour of patients \citep{Bender2015}. This helps to identify patients with DOC better because it demonstrates the level of responsiveness that could not be obtained by the methods previously described \ref{behscales}. 
  
  VS is defined as the absence of self-awareness and the environment \ref{tab: States}. Behaviours are limited to reflexive responses indicating no purposeful movement, neither experience of suffering or evidence of comprehension \citet{multi1994medical}.
  
  %\Mcs
  MCS is serious but does not represent a complete lack of awareness resulting  from widespread damage to the telencephalon (the part of the brain that controls thinking and behaviour).
  The intentionally and types of behaviour exhibited by people in a VS and a MCS,
 can be challenging to distinguish, and subtle signs of consciousness can go unnoticed.
It is widely recognized that the use of standardized and sensitive behavioural assessment scales,
such as the Sensory Modality Assessment and Rehabilitation Technique (SMART), can
help healthcare professionals identify subtle signs of awareness.

With the data gathered from various patient assessments, we think it is possible to train an artificial intelligence model to recognize hidden patterns or minute details resulting from the multiple data that fed the model.
The new data will generalise the upcoming data and produce a diagnosis enough to be of help to SMART assessors. \citep{Yu2018}.



\section{SMART} \label{sec:smart}

\paragraph{}SMART is an assessment tool that combines communication, motor and sensory modalities to diagnose patients who have suffered severe brain injuries \citep{DaConceicaoTeixeira2018}.

Five levels: no response, reflex response, withdrawal response, localized response and differentiated response.

The main advantage of the method is entirely credible and accepted by the healthcare community that deals with this clinical population.


The disadvantage is that it requires and consumes many resources in making the diagnosis.




\espaco




Research continues on clinical tools such as (fMRI) with improved diagnostic certainty and prognostic applications \cite{Bender2015}. There are three main factors that influence the prognosis of patients in VS and MCS:
\begin{itemize}
\item 

Time (the longer in the state, the more complicated functional recovery becomes)

\item Age (young people have a higher recovery rate, linked to physiological recovery processes and brain plasticity)
\item Type (if non-traumatic, there is a shorter potential recovery window)
\item Note: The more severe the degree of injury, the rarer the recovery.
\end{itemize}

Note: The diagnostic evaluation is done over a period of 3 weeks over 10 sessions, distributed in equal numbers in the morning and afternoon.\citep{Gill-Thwaites2004}

\section{Objectives} \label{sec:objectives}
\paragraph{}The goal of this dissertation is to give greater certainty in the diagnosis, or for instance reduce levels of misdiagnosis of DOC. The following steps materialize the aim:
\begin{enumerate}
    \item Classification in 2 possible stages MCS and VS
    \item Reduce time procedures diagnosis: after the first complete goal check, reduce the number of sessions to less than 10
    \item Try to find correlations between origin and possible stages and gauge the accuracy of results with basis on number of sessions available
\end{enumerate}
\espaco



\section{Structure of the document} \label{sec:struct}

\paragraph{}In total, the dissertation has five chapters.
\paragraph{}Chapter 1 contextualizes DOC and SMART methodologies and the advantages of combining them both with artificial intelligence \ref{chap:intro}.
\paragraph{} The chapter 2 contains state of the art, description of the most common diagnostic tools and their impact on the development of medical diagnosis~\ref{chap:sota}.
\paragraph{} Chapter 3 contains machine learning theory, a guide on how to create a model and how to validate the results~\ref{chap:chap3}.
\paragraph{} Chapter 4 talks about experimental studies related to SMART and the database  ~\ref{chap:chap4}.
\paragraph{} Chapter 5 is the conclusions that describe what was done, the results obtained and what there is more to explore ~\ref{chap:concl}.

\espaco \espaco \espaco 




\textbf{Areas: }\\ CCS \textrightarrow   Computing \hspace{0.1cm} methodologies \textrightarrow  Machine\hspace{0.1cm} learning \\ CCS \textrightarrow   Applied \hspace{0.1cm}computing \textrightarrow    Life\hspace{0.1cm} and \hspace{0.1cm}medical \hspace{0.1cm}sciences

 